\documentclass{article}
\usepackage[utf8]{inputenc}
\usepackage[spanish]{babel}
\usepackage{hyperref}

\title{Plataforma microfluídicas capilares.\\Una buena combianción.}
\author {Miguel M Erenas, Fermin Capitán Vallvey e Ignacio de Orbe-Payá\\
Dpto. Química Analítica. Universidad de Granada}


% Hint: \title{what ever}, \author{who care} and \date{when ever} could stand 
% before or after the \begin{document} command 
% BUT the \maketitle command MUST come AFTER the \begin{document} command! 




\begin{document}
	\maketitle
\begin{abstract}
URL repositorio: \url{https://github.com/Obiblion/proyecto_final}\\
Dentro de los dispositivos Point-of-Care (POC), los sensores microfluídicos se caracterizan por utilizar pequeños volúmenes tanto de muestra como de reactivos necesarios para acometer su función. Existen dispositivos microfluídicos que utilizan diferentes materiales como soporte, que los dotan de una serie de características específicas. El material más utilizado en la actualidad es el papel, aunque recientemente se están comenzando a usar nuevos soportes como hilo o tela de algodón.  
\end{abstract}
	
	
	
	

\end{document}



